\documentclass[12pt,a4paper]{article}
\usepackage[margin=2.5cm]{geometry}
\usepackage{hyperref}
\usepackage{booktabs}
\usepackage{xepersian}
\settextfont{XB-Niloofar.ttf}
\setcounter{secnumdepth}{-2}

\title{\textbf{گزارش پروژه درس وب} }
\author{آرمان آریان‌پور \and کوشا معینی \and بهراد مظاهری}
\date{}
\begin{document}
\maketitle

%% ============================================================
\part{آرمان آریان‌پور}
%% ============================================================

\section{مسئولیت‌ها و کارهای انجام شده}

\subsection{فاز بک‌اِند (بخش‌های ۱، ۲ و ۳)}
من زیرساخت احراز هویت و چرخه کامل پرونده‌سازی را طراحی کردم.

\textbf{بخش ۱ — ثبت‌نام و ورود:}
سیستم ثبت‌نام با فیلدهای یکتای نام کاربری، کد ملی، ایمیل و شماره تماس پیاده‌سازی شد.
کاربر می‌تواند با هر یک از این چهار شناسه وارد سیستم شود. نقش‌ها داینامیک بوده و مدیر
می‌تواند هر نقش را به کاربران تخصیص دهد. احراز هویت با \lr{SimpleJWT} و
چرخش خودکار \lr{Refresh Token} ایمن شد.

\textbf{بخش ۲ — تشکیل پرونده:}
دو مسیر ایجاد پرونده: از طریق شکایت (با جریان تایید کارآموز $\to$ افسر پلیس) و از طریق
مشاهده صحنه جرم. قانون ابطال پرونده پس از ۳ اطلاعات ناقص از شاکی پیاده‌سازی شد.

\textbf{بخش ۳ — ثبت شواهد:}
چهار نوع مدل شاهد پیاده‌سازی شد: استشهاد شاهدان، شواهد زیستی (با فیلد نتیجه پزشکی),
وسایل نقلیه (پلاک یا شماره سریال به صورت منع‌الجمع) و مدارک شناسایی (ساختار کلید-مقدار).

\subsection{فاز فرانت‌اِند (بخش‌های ۱، ۲ و ۳)}
\begin{itemize}
    \item \textbf{صفحه اصلی:} معرفی اداره پلیس با سه آمار زنده (پرونده‌های حل‌شده، کارمندان، پرونده‌های فعال).
    \item \textbf{صفحه ورود/ثبت‌نام:} فرم چندمرحله‌ای با اعتبارسنجی کامل سمت کلاینت.
    \item \textbf{داشبورد ماژولار:} ماژول‌ها بر اساس نقش فیلتر می‌شوند (مثلاً کارآگاه تخته کارآگاه می‌بیند اما پزشک قانونی نمی‌بیند).
\end{itemize}

\section{قراردادهای توسعه}
\begin{itemize}
    \item \textbf{نام‌گذاری:} \lr{PascalCase} برای مدل‌های جنگو؛ \lr{snake\_case} برای فیلدهای \lr{JSON} و \lr{API}.
    \item \textbf{پیام‌های کامیت:} الگوی \lr{Conventional Commits}:
    \begin{latin}\small
    \texttt{feat(auth): add JWT refresh token rotation}\\
    \texttt{feat(case): implement 3-strike cancellation logic}\\
    \texttt{feat(evidence): add vehicle plate/serial mutual exclusion}
    \end{latin}
\end{itemize}

\section{نحوه مدیریت پروژه}
وظایفم پیش‌نیاز بقیه بخش‌ها بودند. مدل‌های \lr{User}، \lr{Case} و \lr{Evidence}
را ابتدا به صورت stable تحویل دادم تا سایر اعضا بتوانند کار خود را شروع کنند.
هماهنگی از طریق \lr{GitHub Issues} و جلسات هفتگی انجام می‌شد.

\section{موجودیت‌های کلیدی سامانه}
\begin{center}
\begin{tabular}{@{}lp{9cm}@{}}
\toprule
\textbf{موجودیت} & \textbf{دلیل وجود} \\
\midrule
\lr{CustomUser} & شناسه‌های چندگانه یکتا لازم است که مدل پیش‌فرض جنگو پشتیبانی نمی‌کند \\
\lr{Role} & نقش‌ها داینامیک هستند؛ مدیر باید بدون تغییر کد نقش اضافه/حذف کند \\
\lr{Case} & واحد اصلی فعالیت؛ همه چیز به پرونده وابسته است \\
\lr{CaseRevision} & نگهداری پیام‌های خطای کارآموز برای شفافیت فرآیند \\
\lr{EvidenceKeyValue} & مدارک شناسایی تعداد فیلد مشخصی ندارند؛ ساختار \lr{JSON} لازم است \\
\bottomrule
\end{tabular}
\end{center}



\section{نمونه کد تولید شده توسط هوش مصنوعی}

\textbf{بخش عمده‌ای از استایل‌دهی صفحه داشبورد با Tailwind:}
\begin{latin}
\begin{verbatim}
<div className="min-h-screen bg-gray-950 text-gray-100
               flex flex-col items-start px-8 py-6">
  <header className="w-full flex justify-between items-center
                     border-b border-gray-700 pb-4 mb-6">
    <h1 className="text-2xl font-bold tracking-widest
                  uppercase text-amber-400">
      L.A. Noire — Police Department
    </h1>
    <button className="bg-red-700 hover:bg-red-600 text-white
                       px-4 py-2 rounded text-sm font-medium
                       transition-colors duration-150">
      Logout
    </button>
  </header>

  <main className="grid grid-cols-1 md:grid-cols-3
                  gap-6 w-full">
    <StatCard label="Active Cases" value={stats.active}
      className="bg-gray-800 border border-gray-700
                 rounded-lg p-5 shadow-lg" />
    <StatCard label="Solved Cases" value={stats.solved}
      className="bg-gray-800 border border-gray-700
                 rounded-lg p-5 shadow-lg" />
    <StatCard label="Officers"     value={stats.officers}
      className="bg-gray-800 border border-gray-700
                 rounded-lg p-5 shadow-lg" />
  </main>
</div>
\end{verbatim}
\end{latin}

\textbf{Validator منع‌الجمع پلاک و سریال در بک‌اند:}
\begin{latin}
\begin{verbatim}
def clean(self):
    if self.plate_number and self.serial_number:
        raise ValidationError(
            "Cannot have both plate and serial number."
        )
\end{verbatim}
\end{latin}

\textbf{Interceptor خودکار برای Refresh Token:}
\begin{latin}
\begin{verbatim}
api.interceptors.response.use(
  res => res,
  async err => {
    const original = err.config;
    if (err.response?.status === 401 && !original._retry) {
      original._retry = true;
      const { data } = await axios.post('/auth/token/refresh/',
        { refresh: localStorage.getItem('refresh') });
      api.defaults.headers.common['Authorization'] =
        'Bearer ' + data.access;
      return api(original);
    }
    return Promise.reject(err);
  }
);
\end{verbatim}
\end{latin}

\section{ضعف‌ها و قوت‌های هوش مصنوعی در فرانت‌اِند}
\textbf{قوت:} سرعت نوشتن \lr{CSS} و کلیات صفحات فرانت را به شدت بالا برد. ساختار اولیه
صفحات، لایوت گرید و رنگ‌بندی کلی را در چند دقیقه آماده می‌کرد که در برنامه‌نویسی معمولی
ساعت‌ها وقت می‌گرفت. \\[4pt]
\textbf{ضعف:} ضعف شدید در جزئیات و یکدست نبودن خروجی. مثلاً با دستور «همه دکمه‌ها را
قرمز کن» فقط برخی دکمه‌ها تغییر می‌کردند و بقیه دست‌نخورده می‌ماندند. سلیقه بصری
ضعیفی داشت و بدون راهنمایی دقیق، ظاهر یکدستی تولید نمی‌کرد.

\section{ضعف‌ها و قوت‌های هوش مصنوعی در بک‌اِند}
\textbf{قوت:} ایده‌های خوبی برای ساختار اولیه پروژه داشت؛ از جمله پیشنهاد نحوه تنظیم
\lr{URL}ها و وصل کردن بخش‌های مختلف کد از طریق مسیریابی و \lr{API endpoint}های منطقی. \\[4pt]
\textbf{ضعف:} مشکلات زیادی در برقراری روابط نقش‌ها داشت. تعریف مدل داینامیک \lr{Role}
و تخصیص آن به \lr{CustomUser} را بارها اشتباه پیاده می‌کرد و کدهای تولیدشده اغلب
با معماری \lr{Permission} موجود در \lr{DRF} سازگار نبودند.

\newpage
%% ============================================================
\part{کوشا معینی}
%% ============================================================

\section{مسئولیت‌ها و کارهای انجام شده}

\subsection{فاز بک‌اِند (بخش‌های ۴، ۵ و ۶)}
من هسته تعاملی پروژه را پیاده‌سازی کردم؛ از محیط بصری کارآگاه تا جریان بازجویی و دادگاه.

\textbf{بخش ۴ — حل پرونده (تخته کارآگاه):}
دو موجودیت اصلی تعریف شد: \lr{BoardItem} برای ذخیره مختصات \lr{x,y} هر مدرک روی تخته
و \lr{BoardLink} برای ذخیره جفت‌های متصل. مکانیزم اعلان با \lr{Django Signal} پیاده‌سازی
شد تا کارآگاه بلافاصله پس از ثبت مدرک جدید نوتیف دریافت کند. \lr{API PATCH} برای
بروزرسانی مختصات پس از هر \lr{drag} نیز طراحی شد.

\textbf{بخش ۵ — بازجویی:}
سیستم امتیازدهی دوگانه: گروهبان و کارآگاه هر کدام مستقل عدد ۱ تا ۱۰ می‌دهند و میانگین
به کاپیتان ارسال می‌شود. برای جرایم با درجه بحرانی، تایید نهایی رئیس پلیس هم الزامی است.
مدل \lr{InterrogationScore} برای جدا نگه داشتن نظر هر مقام طراحی شد.

\textbf{بخش ۶ — محاکمه:}
مدل \lr{Trial} یک \lr{Snapshot} کامل از پرونده در لحظه محاکمه نگه می‌دارد تا اگر
مدارک بعداً ویرایش شدند، رای قاضی بر اساس اطلاعات همان لحظه باشد. فیلد \lr{verdict}
و \lr{punishment} نظر نهایی دادگاه را ثبت می‌کنند.

\subsection{فاز فرانت‌اِند (بخش‌های ۴، ۵ و ۶)}
\textbf{تخته کارآگاه:}
هر مدرک به صورت کارت \lr{Draggable} روی یک \lr{canvas div} قرار می‌گیرد. پس از رها کردن
کارت، مختصات جدید با \lr{API PATCH} ذخیره می‌شود. خطوط قرمز با \lr{xarrows} بین
کارت‌های متصل رسم می‌شوند. دکمه خروجی با \lr{html-to-image} کل تخته را به \lr{PNG}
تبدیل و دانلود می‌کند تا قاضی بتواند آن را به پرونده ضمیمه کند.

\textbf{صفحه Most Wanted:}
لیست مظنونین با عکس، نام، مبلغ پاداش و امتیاز رتبه‌بندی نمایش داده می‌شود. مرتب‌سازی
بر اساس فرمول امتیاز است و تازه‌واردهای لیست با برچسب ویژه مشخص می‌شوند.

\textbf{صفحه وضعیت پرونده‌ها:}
بر اساس نقش کاربر، دکمه‌های تایید، رد یا ارسال پیام خطا به کارآموز نمایش داده می‌شود.
کارآموز پیغام خطا دریافت می‌کند، افسر می‌تواند تایید یا رد کند و کارآگاه وضعیت کلی را
مشاهده می‌کند.

\section{قراردادهای توسعه}
\begin{itemize}
    \item \textbf{نام‌گذاری:} \lr{PascalCase} برای کامپوننت‌ها؛ \lr{camelCase} برای توابع.
    \item \textbf{پیام‌های کامیت:}
    \begin{latin}\small
    \texttt{feat(board): add drag-and-drop card positioning}\\
    \texttt{feat(board): implement red line connections with xarrows}\\
    \texttt{fix(trial): fix judge verdict submission endpoint}
    \end{latin}
\end{itemize}

\section{نحوه مدیریت پروژه}
هر بخش را روی یک برنچ جداگانه توسعه دادم. تخته کارآگاه را ابتدا با داده‌های \lr{mock} ساختم
و سپس به \lr{API} متصل کردم. \lr{API contract} مشترک را هم در یک فایل جداگانه نگه داشتیم.

\section{موجودیت‌های کلیدی سامانه}
\begin{center}
\begin{tabular}{@{}lp{9cm}@{}}
\toprule
\textbf{موجودیت} & \textbf{دلیل وجود} \\
\midrule
\lr{BoardItem} & ذخیره مختصات \lr{x,y} مدارک روی تخته برای بازسازی بین جلسات \\
\lr{BoardLink} & ذخیره جفت‌های متصل مدارک برای رسم خطوط در بار بعدی \\
\lr{InterrogationScore} & نظرات گروهبان و کارآگاه مستقل هستند؛ یک مدل واحد نمی‌توان داشت \\
\lr{Trial} & \lr{Snapshot} کامل پرونده در لحظه محاکمه؛ مدارک ممکن است بعداً تغییر کنند \\
\lr{Notification} & کارآگاه باید بلافاصله از مدارک جدید مطلع شود \\
\bottomrule
\end{tabular}
\end{center}



\section{نمونه کد تولید شده توسط هوش مصنوعی}

\textbf{ذخیره مختصات کارت پس از \lr{drag}:}
\begin{latin}
\begin{verbatim}
const handleDragStop = (id, data) => {
  setBoardItems(prev =>
    prev.map(item =>
      item.id === id ? { ...item, x: data.x, y: data.y } : item
    )
  );
  api.patch(`/board-items/${id}/`, { x: data.x, y: data.y });
};
\end{verbatim}
\end{latin}

\textbf{خروجی تصویری از تخته:}
\begin{latin}
\begin{verbatim}
const exportBoard = () => {
  htmlToImage.toPng(boardRef.current, { quality: 0.95 })
    .then(dataUrl => {
      const link = document.createElement('a');
      link.download = `case-board-${caseId}.png`;
      link.href = dataUrl;
      link.click();
    });
};
\end{verbatim}
\end{latin}

\section{ضعف‌ها و قوت‌های هوش مصنوعی در فرانت‌اِند}
\textbf{قوت‌ها:} پیشنهاد پکیج \lr{xarrows} که خودم نمی‌شناختم؛ تولید سریع منطق \lr{Drag-and-Drop}. \\[4pt]
\textbf{ضعف‌ها:} در مدیریت تداخل \lr{z-index} خطوط و کارت‌ها اشتباه می‌کرد؛ کد خروجی
تصویری را با کتابخانه‌های منسوخ می‌نوشت.

\section{ضعف‌ها و قوت‌های هوش مصنوعی در بک‌اِند}
\textbf{قوت‌ها:} طراحی مدل \lr{Trial} با تمام \lr{relation}های لازم؛ نوشتن سریع \lr{Signal}
برای اعلان به کارآگاه. \\[4pt]
\textbf{ضعف‌ها:} \lr{Snapshot} پرونده برای قاضی را عمق کافی نمی‌داد؛ پیاده‌سازی رتبه‌بندی
بازجویی را ساده‌انگارانه می‌نوشت.

\newpage
%% ============================================================
\part{بهراد مظاهری}
%% ============================================================

\section{مسئولیت‌ها و کارهای انجام شده}

\subsection{فاز بک‌اِند (بخش‌های ۷، ۸ و ۹)}
من بخش‌های انتهایی چرخه تجاری را پیاده‌سازی کردم و همچنین مسئول استقرار کل سیستم بودم.

\textbf{بخش ۷ — وضعیت مظنونین (Most Wanted):}
فرمول رتبه‌بندی مظنونین به صورت زیر تعریف شد:
\[ \text{Score} = \max(L_j) \times \max(D_i) \]
که $L_j$ بیشترین روزهای تعقیب در یک پرونده باز و $D_i$ بالاترین درجه جرم ارتکابی است.
فرمول پاداش ریالی: $\max(L_j) \times \max(D_i) \times 20{,}000{,}000$ ریال.
محاسبه روزانه با \lr{Celery Beat} انجام می‌شود و نتیجه در مدل \lr{WantedScore} کش می‌شود
تا درخواست‌های صفحه عمومی بدون بار محاسباتی پاسخ بگیرند.

\textbf{بخش ۸ — گزارش مردمی و پاداش:}
جریان گزارش مردمی: شهروند گزارش می‌دهد، افسر بررسی می‌کند، کارآگاه تایید نهایی می‌دهد.
پس از تایید، یک \lr{UUID} منحصربه‌فرد (\lr{RewardToken}) صادر می‌شود. شهروند می‌تواند
با وارد کردن کد ملی و این شناسه، وضعیت و مبلغ پاداشش را استعلام بگیرد.

\textbf{بخش ۹ — وثیقه (optional):}
مدل \lr{BailPayment} طراحی شد اما پیاده‌سازی کامل به دلیل محدودیت زمانی به فاز بعدی
موکول شد. \lr{API} پایه برای ثبت وثیقه موجود است.

\subsection{فاز فرانت‌اِند (بخش‌های ۷ و ۸)}
\textbf{پنل گزارش‌گیری کلی:}
صفحه‌ای مخصوص قاضی، کاپیتان و رئیس پلیس که تمام پرونده‌های سیستم را با فیلترهای
داینامیک (بازه تاریخ، وضعیت پرونده، نام کارآگاه) نمایش می‌دهد. جزئیات کامل هر پرونده
شامل شواهد، مظنونین، اعضای پلیس دخیل و تاریخچه تغییرات قابل مشاهده است.

\textbf{فرم داینامیک ثبت مدرک:}
فرمی که فیلدهایش بر اساس نوع مدرک انتخابی تغییر می‌کند: برای مدرک زیستی فیلد
نتیجه آزمایش پزشکی ظاهر می‌شود، برای وسیله نقلیه پلاک یا سریال، و برای مدارک شناسایی
یک \lr{key-value editor} داینامیک. این الگو با \lr{useWatch} از \lr{react-hook-form}
بدون رندر بیهوده پیاده‌سازی شد.

\textbf{صفحه استعلام پاداش:}
فرمی عمومی (بدون نیاز به لاگین) که شهروند کد ملی و شناسه یکتا وارد می‌کند و وضعیت
پاداش را مشاهده می‌کند.

\section{قراردادهای توسعه}
\begin{itemize}
    \item \textbf{نام‌گذاری سرویس‌های \lr{Docker}:} \lr{lowercase kebab-case} (\lr{wp-backend}, \lr{wp-db})
    \item \textbf{پیام‌های کامیت:}
    \begin{latin}\small
    \texttt{feat(reward): add unique ID generation for witness}\\
    \texttt{feat(wanted): implement score ranking formula}\\
    \texttt{chore(docker): add nginx reverse proxy config}
    \end{latin}
\end{itemize}

\section{نحوه مدیریت پروژه}
وظایفم در انتهای چرخه تجاری قرار داشت. با \lr{seed data} سایر اعضا تست می‌کردم.
\lr{Dockerfile} هر سرویس توسط عضو مربوطه نوشته شد و من آن‌ها را در \lr{docker-compose.yml}
ادغام کردم.

\section{موجودیت‌های کلیدی سامانه}
\begin{center}
\begin{tabular}{@{}lp{9cm}@{}}
\toprule
\textbf{موجودیت} & \textbf{دلیل وجود} \\
\midrule
\lr{WantedScore} & \lr{Cache} امتیاز برای جلوگیری از محاسبات سنگین در هر \lr{request} \\
\lr{CitizenReport} & از گزارش مردمی تا تایید پلیس چند مرحله دارد؛ نیاز به موجودیت مستقل \\
\lr{RewardToken} & شناسه یکتایی که شهروند با آن پاداش می‌گیرد \\
\lr{RewardClaim} & تاریخچه پرداخت‌ها برای شفافیت مالی \\
\bottomrule
\end{tabular}
\end{center}



\section{نمونه کد تولید شده توسط هوش مصنوعی}

\textbf{فرمول محاسبه پاداش با \lr{ORM} جنگو:}
\begin{latin}
\begin{verbatim}
def calculate_reward(suspect):
    open_cases = suspect.cases.filter(status='OPEN')
    max_days = open_cases.annotate(
        days=ExpressionWrapper(
            Now() - F('opened_at'),
            output_field=DurationField()
        )
    ).aggregate(max_d=Max('days'))['max_d']
    max_severity = suspect.cases.aggregate(
        max_s=Max('crime_level'))['max_s'] or 1
    days_int = max_days.days if max_days else 0
    return days_int * max_severity * 20_000_000
\end{verbatim}
\end{latin}

\textbf{فرم داینامیک ثبت مدرک در \lr{React}:}
\begin{latin}
\begin{verbatim}
const EvidenceForm = ({ caseId }) => {
  const { register, watch } = useForm();
  const evidenceType = watch('type');
  return (
    <form>
      <select {...register('type')}>
        <option value="biological">زیستی</option>
        <option value="vehicle">وسیله نقلیه</option>
      </select>
      {evidenceType === 'vehicle' && <VehicleFields />}
      {evidenceType === 'biological' && <BioFields />}
    </form>
  );
};
\end{verbatim}
\end{latin}

\section{ضعف‌ها و قوت‌های هوش مصنوعی در فرانت‌اِند}
\textbf{قوت‌ها:} تولید سریع جداول گزارش با فیلترهای داینامیک؛ پیشنهاد الگوی \lr{Conditional
Field Rendering} در فرم مدارک. \\[4pt]
\textbf{ضعف‌ها:} در محاسبه مبلغ پاداش سمت فرانت اشتباهات \lr{float precision} داشت؛
پیشنهادهای قدیمی \lr{React Router v5} می‌داد.

\section{ضعف‌ها و قوت‌های هوش مصنوعی در بک‌اِند}
\textbf{قوت‌ها:} طراحی \lr{Celery beat} برای محاسبه روزانه؛ نوشتن کوئری‌های بهینه برای
فرمول رتبه‌بندی با \lr{ORM} جنگو. \\[4pt]
\textbf{ضعف‌ها:} پیکربندی \lr{nginx} در \lr{Docker} به اشتباه \lr{timeout} می‌گذاشت؛
برای محدودیت‌های شبکه ایران راه‌حلی نداشت.

\newpage
%% ============================================================
\part*{بخش مشترک — تیم}
%% ============================================================

\section{پکیج‌های \lr{NPM} استفاده شده در پروژه}
جدول زیر پکیج‌های اصلی مشترک کل پروژه را نشان می‌دهد.

\begin{center}
\begin{tabular}{@{}llp{6.5cm}@{}}
\toprule
\textbf{پکیج} & \textbf{کارکرد} & \textbf{توجیه انتخاب} \\
\midrule
\lr{react-hook-form} & مدیریت فرم‌ها & کمترین رندر بیهوده؛ پشتیبانی از \lr{fieldArray} برای فرم‌های داینامیک \\
\lr{axios} & درخواست \lr{HTTP} & امکان تعریف \lr{Interceptor} برای \lr{Refresh Token} خودکار در یک مکان \\
\lr{zustand} & مدیریت \lr{State} & سبک‌تر از \lr{Redux}؛ بدون \lr{boilerplate}؛ اشتراک \lr{State} بین کامپوننت‌های دور از هم \\
\lr{react-draggable} & \lr{Drag-and-Drop} کارت‌ها & پیاده‌سازی ساده بدون نیاز به \lr{Canvas}؛ مختصات دقیق در \lr{onStop} \\
\lr{xarrows} & رسم خطوط بین \lr{DOM} & تنها پکیج بالغ برای رسم خطوط بین المان‌های معمولی بدون \lr{SVG} دستی \\
\lr{tailwindcss} & استایل‌دهی & توسعه سریع؛ \lr{dark mode} آماده؛ هماهنگی تم \lr{Noir} در کل پروژه \\
\bottomrule
\end{tabular}
\end{center}

\section{نیازسنجی ابتدایی و نهایی پروژه}

\subsection{نیازسنجی ابتدایی}
در ابتدای پروژه، سیستم به صورت یک اپلیکیشن ساده تعریف شده بود:
\begin{itemize}
    \item ثبت‌نام و ورود با ایمیل و رمز عبور
    \item یک جدول کاربر با نقش‌های ثابت (پلیس، مدیر)
    \item فرم ساده ثبت پرونده جنایی
    \item لیست مجرمین تحت تعقیب
\end{itemize}

\subsection{نیازسنجی نهایی}
پس از بررسی مستندات سیستم واقعی، نیازمندی‌ها به شکل زیر گسترش یافتند:
\begin{itemize}
    \item احراز هویت چندشناسه‌ای (نام کاربری، کد ملی، ایمیل، شماره تماس) با \lr{JWT}
    \item سیستم نقش داینامیک؛ مدیر بدون تغییر کد نقش جدید اضافه می‌کند
    \item دو مسیر تشکیل پرونده با جریان تایید کارآموز $\to$ افسر
    \item چهار نوع مدل شاهد با اعتبارسنجی‌های تخصصی
    \item تخته کارآگاه بصری با \lr{Drag-and-Drop} و خطوط قرمز و خروجی \lr{PNG}
    \item بازجویی با امتیازدهی مستقل چند مقام
    \item فرمول رتبه‌بندی ریاضی مظنونین با محاسبه روزانه
    \item سیستم گزارش مردمی و پاداش با شناسه یکتا
    \item استقرار کامل با \lr{Docker Compose} و \lr{nginx}
\end{itemize}

\subsection{قوت‌های تصمیمات گرفته شده}
\begin{itemize}
    \item \textbf{نقش داینامیک:} مدیر سیستم بدون دخالت توسعه‌دهنده نقش جدید اضافه یا حذف می‌کند؛ انعطاف بالا برای توسعه آینده.
    \item \textbf{\lr{JSONField} برای مدارک شناسایی:} تعداد فیلدهای مدارک از پیش مشخص نیست؛ ساختار کلید-مقدار این مشکل را بدون \lr{migration} جدید حل می‌کند.
    \item \textbf{ذخیره مختصات در \lr{DB}:} تخته کارآگاه بین جلسات مختلف و کاربران مختلف یکسان باقی می‌ماند.
    \item \textbf{\lr{WantedScore Cache}:} صفحه عمومی مظنونین بدون محاسبه سنگین در هر درخواست پاسخ می‌دهد.
    \item \textbf{\lr{Docker Compose}:} محیط توسعه هر سه نفر یکسان شد و مشکل «روی ماشین من کار می‌کند» از بین رفت.
\end{itemize}

\subsection{ضعف‌های تصمیمات گرفته شده}
\begin{itemize}
    \item \textbf{پیچیدگی جریان تایید پرونده:} زنجیر کارآموز $\to$ افسر $\to$ کارآگاه بیشتر از برنامه زمان گرفت و \lr{edge case}های آن کامل پوشش داده نشد.
    \item \textbf{\lr{Polling} به جای \lr{WebSocket}:} اعلان مدارک جدید به کارآگاه با \lr{polling} پیاده شد که بار شبکه‌ای ایجاد می‌کند.
    \item \textbf{\lr{Celery} در استقرار:} اضافه شدن \lr{Celery} و \lr{Redis} به \lr{docker-compose} پیچیدگی \lr{deployment} را بالا برد.
    \item \textbf{بخش وثیقه ناتمام:} بخش ۹ (پرداخت وثیقه) به دلیل محدودیت زمانی کامل پیاده‌سازی نشد.
\end{itemize}

\end{document}
